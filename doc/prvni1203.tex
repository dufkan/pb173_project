\documentclass{beamer}
%
% Choose how your presentation looks.
%
% For more themes, color themes and font themes, see:
% http://deic.uab.es/~iblanes/beamer_gallery/index_by_theme.html
%
\mode<presentation>
{
	\usetheme{default}      % or try Darmstadt, Madrid, Warsaw, ...
	\usecolortheme{default} % or try albatross, beaver, crane, ...
	\usefonttheme{default}  % or try serif, structurebold, ...
	\setbeamertemplate{navigation symbols}{}
	\setbeamertemplate{caption}[numbered]
} 

%\usepackage[english]{babel}
\usepackage[shorthands=off,czech]{babel}
\usepackage[utf8x]{inputenc}

\title[Your Short Title]{CryMe}
\subtitle{PB173 Projekt}
\author{Antonín Dufka, Štěpánka Gennertová}
%\institute{Where You're From}
\date{12.3.2018}

\begin{document}
	
	\begin{frame}
	\titlepage
\end{frame}



\section{Úvod}
\begin{frame}{Úvod}
	Aplikace CryMe bude sloužit k výměně zašifrovaných zpráv mezi uživateli skrze centrální server. Síťová komunikace bude probíhat skrze TCP spojení iniciovaná klienty. 
\end{frame}

\begin{frame}{Klient}
Možnosti uživatele
\begin{itemize}
	\item vygenerování dvojice klíčů pro asymetrickou komunikaci
	\item zaregistrování se do aplikace na základě veřejného klíče a pseudonymu
	\item přihlášení se k serveru
	\item naváyání spojení s jiným uživatelem
	\item šifrovaná komunikace s jiným uživatelem (výměna zpráv)
	\item ukončení spojení se serverem
	\item pamatování si uživatelů, se kterými se už spojil
\end{itemize}
\end{frame}


\begin{frame}{Kryptografie}
K zajištění bezpečnosti využíváme
\begin{itemize}
	\item RSA-2048 pro asymetrickou kryptografii (při iniciální komunikaci mezi serverem a klientem)
	\item challenge - response protokol pro autentizaci klienta i serveru (na jehož základě se vygeneruje klíč pro následovnou symetricky šifrovanou komunikaci)
	\item SHA2-256 pro pro vytvoření symetrického klíče (pro každé spojení se generuje nový)
	\item další komunikace mezi klientem a serverem je šifrována AES-256
\end{itemize}
\end{frame}


\begin{frame}{Server}
Funkcionalita serveru
\begin{itemize}
	\item zaregistrování uživatele - jeho pseudonymu a veřejného klíče
	\item drží si databázi pseudonymů a veřejných klíčů uživatelů (příp. i symetrického klíče)
	\item autentizace uživatele
	\item poskytnutí uživateli veřejný klíč jiného uživatele podle zadaného pseudonymu
	\item přeposílání zpráv mezi uživateli
	\item ukončení TCP spojení s uživatelem
\end{itemize}
\end{frame}


\end{document}